\clearpage

\part{Dziecko Ziemi + Las tajemnic / Mroczne posągi}

\begin{twocolumn}

\section{Wprowadzenie}

\DndDropCapLine{C}{zasem posiadanie dodatkowego} scenariusza przystosowanego do aktualnej kampanii może nie rozwijać
głównej linii fabularnej, ale posłużyć za side-quest - można go wykorzystać jako jednostrzałowiec w sytuacji, gdy nie
mieliśmy jako mistrz/mistrzyni gry czasu się przygotować, albo nagle połowa naszej drużyny ma chore dziecko, wizytę w
szpitalu, pracę na następny dzień, zostawiło jutro wieczorem akurat włączone żelazko.

Zamiast przekładać sesję, możemy rozegrać taki "doklejany" scenariusz bez wyrywania graczy z immersji danej kampanii, a
może on nawet wprowadzić ciekawe rozbudowanie bohaterów dalszego planu - pokazać głębię świata, że ten żyje również poza
wydarzeniami bezpośrednio śledzonymi przez poszukiwaczy przygód.

Ja w taki sposób wykorzystałem \emph{Dziecko Ziemi}\cite{dziecko_ziemi} w trakcie rozgrywania \emph{Mrocznych posągów}
\cite{mroczne_posagi} do rozbudowania historii wiedźmy \emph{Esme}. Postacie mogą trafić na polanę jeszcze w trakcie
rozgrywania \emph{Lasu tajemnic}\cite{las_tajemnic}, więc można przygotować sobie ten scenariusz wcześniej.

\section{Opis historii}

% TODO

\section{Punkty styku}

% TODO: Elenya coś wie, drwale w wiosce gadają,

Po odpoczynku Eskiel i Bunron nadal nie czuli się dobrze i nie byli w stanie kontynuować marszu. Jednak na polanę właśnie dotarł prowadzony przez Lalalnę Da’kber, a także Dorgrim, który najwyraźniej również wiedział jak trafić w to miejsce, choć nie wiadomo czego się tutaj spodziewał.
Usłyszał on plotkę od drwali krzątających się po bitwie w wiosce i postanowił odszukać polanę, o której rozmawiali na własną rękę zamiast kierować się za drużyną.
Mimo iż Da’kberowi i Dorgrimowi udało się pokonać wiedźmę po ciężkiej walce, śmierć poniósł Stevo. Wiedźma w chwili śmierci wysyczała jedno słowo – “mamo”.
Rozglądając się po pomieszczeniu zauważyli półkę z trzema niezwykłymi drewnianymi przedmiotami przypominającymi odpowiednio miecz, drzewiec łuku i tarczę. Przedmioty zdawały się w magiczny sposób unosić nad półką. Da’kber chcąc zbadać dokładniej jeden z nich złapał ten, który przypominał drzewiec do łuku.
W momencie, w którym złapał przedmiot cała trójka (Stevo, Da’kber i Dorgrim) przenieśli się nagle do gospody w wiosce, pod wielkim drzewem. Przysiadł się do nich młody jegomość proszący o odnalezienie córki. Przedstawił się imieniem Mandyk, a jego córka – 	 – miała zostać porwana w nocy przez jakąś kobietę.
Bohaterowie postanowili zbadać sprawę i znaleźli ślady konia prowadzące od domu Mandyka na zachód od wioski.
Podążyli po śladach i dotarli do starych elfickich ruin, w których spotkali gobliny, a po ich pokonaniu, w głównym holu spotkali elfią druidkę imieniem Elenya.
Druidka powiedziała, że o dziecko upomniała się natura i jest teraz z matką. Drużyna postanowiła związać Elenyę i odprowadzić do wioski na przesłuchanie.
W wiosce okazało się, że Elenya jest miejscową kapłanką, opiekującą się wioską i Mandyk ją zna. Wyszło na jaw, że przeszłość i pochodzenie dziecka nie są do końca jasne, jednak Elenya wyjawiła miejsce, w którym znajdować się miała jaskinia, do której zabrano dziecko.
Mandyk powiedział, że najpewniej jest tam ze swoją matką, Izmą.
Po drodze na miejsce, w lesie, drużyna została zaatakowana przez diabła z piekieł, jednak mimo ciężkiej walki udało się go pokonać.
Po dotarciu do celu poszukiwacze przygód od razu natknęli się na dziewczynkę, która na ich widok uciekła w głąb jaskini.
Wewnątrz znaleźli trójkę postaci – dziewczynkę Esme, jej matkę w formie błotnego stwora, oraz zbrojnego mężczyzny, od którego Stevo wyczuł zło.
W walce udało się pokonać złego łowcę i w momencie jego śmierci błotnista postać zdążyła jedynie powiedzieć, że “jej córki nie da się uratować”. Chwilę potem Dorgrim, Da’kber i Stevo, który stał cały i zdrowy, znaleźli się z powrotem w pokoju pod grzybową polaną. Da’kber zamiast drewnianej figurki dzierżył w ręku łuk z dziwnymi złotymi inskrypcjami, a pozostałe dwa przedmioty zniknęły.





\end{twocolumn}

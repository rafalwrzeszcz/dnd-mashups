\clearpage

\part{Kłopoty z goblinami + Krew na szlaku}

\begin{twocolumn}

\section{Wprowadzenie}

\DndDropCapLine{K}{łopoty z goblinami\cite{klopoty_z_goblinami} to pierwsza} część znanej w polskiej społeczności D\&D
trylogii \emph{Powrotu Czarnoksiężnika}. Z kolei \emph{Krew na szlaku}\cite{krew_na_szlaku} to mała, samodzielna
przygoda wydana oryginalnie w języku angielskim i przetłumaczona na polski przez Janka Sielickiego - autora wspomnianej
trylogii. Sam autor w opisie tłumaczenia daje wskazówkę, że może ona posłużyć jako wstęp do jego (dłuższej) historii.
Obydwie przygody wiele łączy - dzieją się w leśnej scenerii, głównymi przeciwnikami są gobliny, są przystosowane do
niskopoziomowych postaci, obie mają bliżej nieokreślone miejsce i setting, zatem można je dowolnie umiejscowić.

Ja swoją kampanię ulokowałem w \emph{Zapomnianych Krainach}, więc od początku chciałem wskazać konkretne miejsce - padło
na \textbf{Vilhon Reach} (okolice \textbf{Morza Spadających Gwiazd}), las \textbf{Winterwood}. Nie ma to w tym przypadku
znaczenia, lecz zdefiniowanie lokalizacji da możliwość w rozwijaniu niektórych postaci - na przykład u mnie
\emph{Bartolomeo Bandałyk} był obwoźnym kupcem, który po całej trylogii zaproponował bohaterom zadanie eskortowania go
do rodzinnego \textbf{Cormyru}.

Choć \emph{Kłopoty z Goblinami} są napisane w bardzo przystępny sposób i często prowadzą Mistrza/Mistrzynię Podziemi za
rękę, to dla początkujących prowadzących mogą być pułapką, gdyż są wprowadzeniem do dużo dłuższego ciągu. Ja bardzo się
cieszę z rozegrania najpierw \emph{Krwi na Szlaku}, gdyż była to pierwsza sesja naszej drużyny, pierwsze zetknięcie z
D\&D dla wielu z graczy, a także dla mnie pierwsze prowadzenie sesji od lat - poprzednim razem grałem i prowadziłem gry
jakieś dwadzieścia lat wcześniej i były to czasy edycji 3/3.5.

Pomysł rozegrania tej krótkiej przygody na początek ma kilka zalet:

\begin{itemize}
\item mini przygoda pozwoli zapoznać się z mechanikami, szczególnie walki i wprowadzi atmosferę, a mała ilość bohaterów
    niezależnych nie przytłoczy prowadzącego;
\item potknięcia i przeoczenia nie wpłyną na dalszą część kampanii - przed przystąpieniem do \emph{Kłopotów z Goblinami}
    będziesz wiedzieć na co zwracać więcej uwagi;
\item jest dużo szybsza w przygotowaniu, więc można po nią sięgnąć w ramach upewnienia się, że DeDeKi to to, co chcecie
    robić wieczorami przez najbliższe kilka lat zanim poświęcicie na przygotowania wiele godzin…
\item …albo jeśli właśnie o spędzeniu tych kilku godzin zapomnieliście i macie na wieczór tego dnia coś przygotować i
    chcieliście trylogię Janka Sielickiego, ale już nie zdążycie - \emph{Krew na Szlaku} sprawdzi się nawet "na szybko".
\end{itemize}

\section{Punkty styku}

Dzięki bardzo podobnej scenerii obydwie przygody komponują się bardzo łatwo, a dodatkowo jest kilka detali, które
pomagają je spajać.

\begin{itemize}
\item Obydwie lokacje (wieś \textbf{Stara Dąbrowa} i \textbf{Przydatnik Podróżnika}) mogą być blisko siebie - ja dla
    lepszej izolacji umiejscowiłem je dzień drogi od siebie. Jeśli nie zależy ci na takim oddzieleniu, możesz przenieść
    \textbf{Przydatnik Podróżnika} do wsi, ale wtedy musisz zauważyć, że w mieście będą dwie gospody, a ta przy starym
    drzewie jest istotna dla fabuły i innych pobocznych zadań.
\item Młodzi poszukiwacze przygód, którzy kilka dni temu wynajęli konie i nie wrócili to dzieciaki ze wsi. Możesz dodać
    ich jako nowych mieszkańców, albo zastąpić nimi \emph{Zośkę} i jej bandę.
\item Gobliny, które je porwały pochodzą z plemienia, które niepokoi wioskę.
\end{itemize}

\section{Przydatnik Podróżnika}

Jeśli nie masz pomysłu na rozpoczęcie przygody, stary dobry motyw spotkania w gospodzie zawsze się sprawdzi:

\begin{DndReadAloud}
Każde z was przemierzało las podążając szlakiem prowadzącym na zachód, przez małą miejscowość o nazwie Stara Dąbrowa.
Jednak, ponieważ jest to praktycznie zapadła dziura pośrodku niczego, droga bardzo wam się dłużyła. Dlatego też, jak
niemal każdy z nielicznych podróżnych zapuszczających się w te strony, postanowiliście zatrzymać się w zajeździe
„Przydatnik Podróżnika”.
\end{DndReadAloud}

Następnie kontynuuj według oryginalnych opisów.

Postacie graczy nie muszą się znać - teraz jest czas, kiedy mogą się wzajemnie przedstawić. \emph{Bale} może powiedzieć
graczom, że \textbf{Stara Dąbrowa} znajduje się niecały dzień drogi na zachód. Prowadzi tam w miarę prosta droga. Na tym
trakcie drużyna spotka potem \emph{Yuriego}.

Opisując młodych poszukiwaczy przygód, którzy porzyczyli konie podkreśl ich młody wiek. Niech będzie jasne, że są to
raczej dzieciaki - mogą ich szukać rodzice. Wioska leży niedaleko, także \emph{Bale} może ich znać.

\subsection{Po powrocie}

Nie musisz zdradzać drużynie powiązania z problemami w wiosce - możesz ich tam pokierować i sprawić, aby samo powiązanie
było subtelne, albo żeby sami je odkryli. Jednak \textbf{Przydatnik Podróżnika} i \textbf{Starą Dąbrowę} dzieli
niewielki dystans, z pewnością żyją w jakiejś symbiozie i problemy jednych są znane drugim. Po rozliczeniu się z
\emph{Balem Farnsworthem} ten może przekazać informacje drużynie:

\begin{DndReadAloud}
Po rozliczeniu się z wami Bale dodaje:

– Gobliny zawsze pałętały się po tej okolicy, ale ostatnio stały się niezwykle aktywne i… zorganizowane – jak na takie
kreatury. Coś nimi kieruje?! Tutaj jeszcze nie jest tak źle, ale jeśli będziecie dalej podążać tą drogą, jeszcze tego
samego dnia powinniście dotrzeć do Starej Dąbrowy. Słyszałem, że tam też mają z nimi nie małe problemy.
\end{DndReadAloud}

Drużyna może odeskortować dzieciaki do wioski, ale jeśli będzie to dla nich ciężarem \emph{Bale} zaproponuje, że
zaopiekuje się nimi aż nie wydobrzeją, a następnie odstawi do wioski, bo itak będzie musiał się udać po zapasy.

\section{Obóz goblinów}

Po stoczeniu walki zwróć uwagę graczy na szczególny element wspólny u wszystkich goblinów, wskazujący na przynależność
do klanu Zębogębców (drużyna nie zna tej nazwy, chodzi jedynie o spostrzeżenie):

\begin{DndReadAloud}
Kiedy już opadł bitewny kurz zauważacie zastanawiający szczegół. To tylko drobny detal, ale jenak przykuwa waszą uwagę -
każdy z goblinów miał jakąś osobliwą ozdobę wykonaną z zębów i to najprawdopodobniej ludzkich.
\end{DndReadAloud}

Możesz dodać opis, który podpowie drużynie, że jest to zalążek dalszych zmagań:

\begin{DndReadAloud}
Pokonanie tego niewielkiego oddziału goblinów było niecodziennym doświadczeniem. I nie chodzi o to, że niecodziennie
widujecie gobliny – te tutaj zachowywały się niecodziennie nawet dla kogoś kto widywałby gobliny codziennie. Miały
emblemat klanu, ale nie było w obozie wodza. Miały wyraźne zadania, jednak nie była to ich stała wioska, a jedynie
tymczasowy obóz. Ewidentnie grupka, która się tutaj zapuściła była tylko częścią większej zgrai. Coś wam mówi, że to nie
są ostatnie goblony jakie spotkacie na swojej drodze w tej okolicy.
\end{DndReadAloud}

\section{Stara Dąbrowa}

Uratowanie nastolatków porwanych przez gobliny może być zadaniem pobocznym w wiosce, jeśli gracze nie uratowali ich
wcześniej.

Niezależnie od tego kiedy to nastąpi, na pewno rodzice będą bardzo wdzięczni za uratowanie swoich dzieci. Miej to na
uwadze w razie rozpatrywania relacji między drużyną, a mieszkańcami wsi (albo - jak w moim przypadku - łagodzeniu
konflików: każdemu wszak zdarza się czasem wypalić coś o "wieśniakach").

\end{twocolumn}

\documentclass[letterpaper,twoside,twocolumn,openany,nodeprecatedcode]{dndbook}
\usepackage[english]{babel}
\usepackage[utf8]{inputenc}
\usepackage[singlelinecheck=false]{caption}
\usepackage{subfiles}
\usepackage{listings}
\usepackage{shortvrb}
\usepackage{stfloats}
\usepackage[colorlinks=true,urlcolor=blue,linkcolor=brown]{hyperref}
\usepackage{perpage}
\usepackage{parskip}
\usepackage[backend=biber]{biblatex}

\addbibresource{index.bib}

\captionsetup[table]{labelformat=empty,font={sf,sc,bf,},skip=0pt}

\MakeShortVerb{|}

\lstset{%
    basicstyle=\ttfamily,
    language=[LaTeX]{TeX},
    breaklines=true,
}

\newcommand{\bookauthor}{Rafa\l\ Wrzeszcz}
\newcommand{\booktitle}{D\&D Mashups}
\newcommand{\booksubtitle}{Pomysły na cudze pomysły}

\pdfinfo{
	/Author (\bookauthor)
	/Title (\booktitle)
}

\title{\booktitle \\
\large \booksubtitle}
\author{\bookauthor}

\begin{document}

% Polish translation
\renewcommand*\contentsname{Spis treści}
\renewcommand{\partname}{Część}

\frontmatter

% Title page
% This is "heavily inspired" by https://github.com/Vladar4/DND-5e-LaTeX-Template-Module-Example

\begin{titlepage}
\begin{onecolumn}
\begin{center}
	\vspace{0.5cm}
	{\Huge \booktitle}

	\vspace{0.5cm}
    % Maybe some day some graphics here
	\begin{picture}(500,200)
		\put(0,0){\framebox(500,200)}
	\end{picture}
	%\includegraphics[width=\textwidth]{img/cover.jpg}

	\vspace{0.5cm}
	Jest to projekt zawierający pomysły na fabularne połączenie różnych gotowych (dostępnych drukiem, online albo w
	innej formie) przygód. Jeśli chcesz dodać coś od siebie, możesz to zrobić na
	\href{https://github.com/rafalwrzeszcz/dnd-mashups}{GitHubie}.

	\vspace{0.5cm}
	{\Large \booksubtitle}

	\vfill

	{\Large by \bookauthor}
	\vspace{0.35cm} \\
	Ver: \gitrev

	\vspace{0.35cm}
	\includegraphics[width=0.25\textwidth]{img/dmsguild.jpg}
\end{center}

\begin{minipage}{0.94\textwidth}
{\footnotesize
	DUNGEONS \& DRAGONS, D\&D, Wizards of the Coast, Forgotten Realms, Ravenloft, the dragon ampersand, and all other
	Wizards of the Coast product names, and their respective logos are trademarks of Wizards of the Coast in the USA and
	other countries.\\
	This work contains material that is copyright Wizards of the Coast and/or other authors. Such material is used with
	permission under the Community Content Agreement for Dungeon Masters Guild.\\
	All other original material in this work is copyright by \bookauthor\ and published under the Community Content
	Agreement for Dungeon Masters Guild.}
\end{minipage}
\end{onecolumn}
\end{titlepage}
\clearpage

\tableofcontents

\mainmatter

\clearpage

\begin{onecolumn}
\section{Od autora}
Prowadząc sesje D\&D często korzystam z gotowych przygód i innych materiałów, ale zazwyczaj wplatam je w dłuższą
kampanię łącząc wątki pomiędzy nimi tak, aby nie były to tylko oddzielne etapy, lecz tworzyły spójną całość i zazębiały
się. Pozwala to graczom mocniej zaangażować się, a także sprawia, że ich decyzje mogą mieć większą wagę dając
konsekwencje w późniejszych momentach gry.
\par
Postanowiłem je spisać i udostępnić - je za darmo. Pamiętaj czytelniku, że są to jedynie pomysły na połączenie już
istniejących przygód. Materiały poszczególnych przygód są samodzielnymi utworami - musisz je osobno zakupić (niektóre z
nich są dostępne za darmo), a także każdy z tych otworów może mieć własne postanowienia licencyjne.
\par
Poszczególne pomysły zazwyczaj składają się z kilku aspektów, które można stosować wybiórczo.
\par
Musisz też wiedzieć, że nie jestem nader doświadczonym Mistrzem Gry - gramy zawsze w grupie znajomych i przede wszystkim
dobrze się bawimy. Tak więc nie zdziw się, jeśli niniejszy materiał lub jakiś jego element nie spełni twoich
wygórowanych standardów. Myślę, że jest to uczciwy układ w przypadku ceny zero. Tak, jeśli jesteś z Poznania na przykład
możesz pobrać dwa.
\par
Sam projekt jest udostępniony na platformie \href{https://github.com/rafalwrzeszcz/dnd-mashups}{GitHub}, gdzie każdy
chcący coś poprawić, albo dodać coś nowego, może to zrobić (tak, też za darmo).
\end{onecolumn}

% TODO: PL translation

\clearpage

\part{Kłopoty z Goblinami + Krew na Szlaku}

\chapter{Wprowadzenie}

% TODO
Coś tam coś tam

\clearpage

\part{Klątwa Strahda + Krwawa Mgła + Opowieści z Cormyru - Festyn w Lesie Hullack + Zgrozy: Ogrody Strahda}

\chapter{Wprowadzenie}

% TODO
Coś tam coś tam

\clearpage

\part{Dziecko Ziemi + Las Tajemnic / Mroczne Posągi}

\chapter{Wprowadzenie}

% TODO
Coś tam coś tam


\clearpage
\printbibliography[heading=bibintoc,title=Odnośniki]

\end{document}
